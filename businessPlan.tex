\section{Business Plan}

\subsection{Target Market Definition} % Fixed
The target market for the entire PICA system comprises both electricity producers and electricity consumers, as set forth by the nature of the subsystems. As the power companies supply and own the electricity meters attached to the buildings to which they supply power, the PICA E-meter appeals only to the market of electricity-producing companies. The other two subsystems, the solid-state breakers and base station, target the power-consuming audience, as the devices will assist in monitoring power flow inside the building, where the power company has no presence. As these two markets are essentially exclusive in both membership and interest in the PICA subsystems, the E-meter will be able to function independently of the other consumer-targeted subsystems, and vice versa.

\subsubsection{Power Companies} % Fixed
As power companies currently distribute the whole-building metering hardware that determines how much energy their customer used, the E-meter clearly targets power companies. In fact, the power companies own the power-measuring hardware external to the buildings to while they provide power, so only they may replace or upgrade those devices. At present, power companies send trained meter-readers to read the data from most traditional power meters under their control. The PICA E-meter subsystem aims to improve on this process by automatically sending the measurements to the power company using a means and protocol selected by the particular company. While this will require some hardware customization for each company, the volume of company-specific production should allow the cost to develop the design to spread into a small per-unit cost.

The PICA E-meter subsystem also provides numerous more measures of power than the simple spinning-dial meters. For example, the E-meter will measure the frequency and the RMS voltage of the incoming supply lines, which help indicate the overall quality of power delivered to the customers in the area. This information may also help diagnose any observed issues with power delivery without dispatching a worker to take measurements by hand. In this way, power companies using the PICA E-meter can improve the quality of the service they provide and can save on the labor costs associated with making a site visit.

\subsubsection{Power Consumers}
Although the power company's customers cannot modify the metering panel installed by the power company, they are free to modify the other power distribution components inside their own buildings. The solid-state breakers fall into this category, and provide previously unavailable measurements regarding power consumption and its location within the building. However, as these breakers will replace the pre-existing breakers inside the building, the consumer must be convinced that using the PICA system is worth the trouble and cost of replacing the mechanical circuit breakers with the more feature-rich PICA breakers. To this effect, the most receptive market for the solid-state breakers includes homeowners and building managers who are curious or concerned about power usage inside their building. That is, the people for whom this information can inspire a meaningful change in practice will likely become the first adopters of the subsystem.

The product may also gain a following as an alternative to mechanical breakers in during the construction of a new home or building. This would likely require that the product already have a proven history of reliability and safety, so the previous group of cost- or environmentally-concerned individuals might have to adopt the produce first. If the PICA solid-state breakers become an alternative during construction, the net cost to the user will be lessened, as the building-to-be will not have any pre-existing breakers to discard or replace.

The base station may apply to either of these two consumer groups, as its primary purpose is to manage and interface with the other systems. It does not specifically require the solid-state breakers or the E-meter, but provides little value in a building without any installed PICA systems. The base station exists solely to manage and collect data from other PICA subsystems, as well as format and display these measurements, so its target audience consists of power consumers whose buildings contain at least one of the E-meter or solid-state breakers.

\subsubsection{Market Coordination}
As the E-meter subsystem caters exclusively to power companies while the other subsystems target power consumers, the complete PICA system has no clearly-defined market; neither of the two markets involved desires the entire system. Despite this separation, a clever marketing strategy could motivate one market to pressure the other.

In one scenario, a power company installs the PICA E-meter onto a select set of their customers' buildings. On its own, this should be a transparent change to the consumer. However, marketing the base station as a way to ``see what the power company sees'' about the power delivered could motivate some of the power-consuming market to purchase a PICA subsystem. Giving the power-consuming market a feeling of empowerment or equality to the power company could therefore motivate more consumer-side subsystem sales.

Conversely, the interests of the power-consuming market could generate interest from the power-producing market. In particular, a consumer who already owns the PICA solid-state breakers may appeal to the power company to provide the PICA E-meter: it would expand the amount of information available and give an accurate sense for the upcoming utility bill. Even without any PICA subsystems, the consumer could prod the power company for a PICA E-meter because it would give the power company more information on power quality, which could in turn increase the quality of service provided to the consumer. In these scenarios, the desires of the power-consumer market could influence demand in the power-producing market.

\subsection{Target Market Research} % Fixed
From the team's visit to Consumer's Energy in Jackson Michigan, power companies are very interested in smart home energy meters, such as PICA provides with the E-meter. The team presented a features list to the representative present. who affirmed that the features now included in the E-meter will be valuable to the power companies. However, power companies are already researching and testing smart meter prototypes and samples, so the E-meter will arrive fairly late relative to its competition. Still, of the thousands of power companies in the United States\cite{EIA_Intro}, the PICA system will surely prove interesting to others who may not have considered alternatives. Fewer than half of the homes expected to receive the smart-meter conversion have been converted to date, so the E-meter is still viable in the power company market\cite{Gtech_Smart_Meters}.

The remainder of the system, the base station and the solid-state breakers, will compete in a market of power consumers. The devices currently face a  market of 6,400,000 customers, according to an estimate from the project's Business 396 companion team. Of these customers, approximately 0.1\% may be interested in PICA, giving 64,000 expected sales\cite{Gtech_Renew}. Additionally, despite the recent economic downturn, housing spending has experienced an upward trend\cite{Economic_Predictions}, which may indicate a possible increase in the number of houses that could be constructed with the PICA solid-state breaker technology. The market for the consumer-oriented subsystems seems to be healthy, and even growing, despite the recent economic slump.

\subsection{Feasibility Study}
\subsubsection{Technical Feasibility} % This has been fixed
The goal of the PICA project is to design and produce a unified system to measure power consumption information and deliver it to the user. In order to judge the technical feasibility of the entire system, one must first judge each of the individual subsystems and then the connections between them.

The E-meter provides similar functionality to existing smart meters, which are demonstrated to be technically feasible. The feasibility of producing one under such a time constraint, however, has not been proven. By using pre-existing components whose purposes are to function as a part of a smart meter, the design team should be able to limit the component-level design to providing the connections between the selected components. Although this task will still require a substantial amount of time and design work, it should be feasible to complete before the end of the spring semester.

The solid-state circuit-breaker subsystem provides functionality that is not as commercially available as smart-meter functionality. In this case, the feasibility of the system cannot be implied by the existence of other products. Although the power-monitoring aspect of the breakers can be constructed from existing parts, the circuit-interrupting behavior is not as easy to duplicate. The difficulties associated with this aspect of the subsystem depend on the hardware selected to interrupt the circuit, but the primary concern at present involves the controllability of the circuits. Using \ac{FET} hardware to choke the flow of current has been proven to be possible, but the control signal associated with maintaining the flow of current must stay within a certain range of the power-line voltage, which behaves as a $120 \volt_{RMS}$ sinusoid: a varying control line could not be easily implemented using logic-level voltages. If thyristors provide the control hardware, then a constant  \ac{DC} control signal could allow current to flow, but they may not truly shut off for several milliseconds after the control line directs current to stop.

The base station functions largely as a hub for the PICA system's measurements, and accordingly formats and that information for display to the user. Devices of a similar nature already exist: networking switches and routers routinely redirect data to other devices, and have done so reliably even when multiple communications media are involved. The similarity of the base station to these routers should indicate that the subsystem is indeed technically feasible, and may even be feasible to the project schedule. If the power consumption of the base station will be comparable to that of these networking devices, the base station should consume approximately 10 watts, which could feasibly be an acceptable overhead for most users.

The communication between these subsystems completes the PICA system, and should prove economically and time-feasible if it employs a pre-existing standard. Reliability provides the key indication of the feasibility of a given communications method, and it encompasses issues such as range, error tolerance, and bandwidth. Power consumption is also an important factor, but may be minimized if the selected method fits the previous criteria. 

\subsubsection{Market Feasibility} % This has been fixed
In order for any product to succeed commercially, its perceived value must meet or exceed the price the customer would pay for it. If the PICA system as a whole is to be a feasible market success, it must surpass its competitors in providing value per price. From a practical standpoint, this involves either selling a comparable product for a lower price than the competition, or producing a superior product at a similar price. The PICA project generally aims to follow the second of these two paths. The solid-state circuit breakers include solid-station circuit breakers and circuit-by-circuit power monitoring, both of which seem to be unusual or even unique features, which in turn means that its feasibility depends on the value of its features rather than a lower price. The base station may be viewed as an accessory to the other subsystems, but its function as the output of the collected information gives it a very high value to anyone who desires the information collected by the other PICA devices, so its feasibility relies on its high perceived value, rather than on undercutting competition. The smart-metering aspects of the E-meter essentially meet the expectations set by other smart meters, so the market feasibility of the E-meter device depends more on the price than on the features, but its ties with the base station can provide additional feature value as well. Overall, the subsystems of the PICA project tend to focus on providing valuable features, rather than on reducing the price below that of the competitors.

\subsubsection{Legal Feasibility} % This has been fixed
The PICA system must meet certain codes in order to be safe enough for the customer to use, which will also protect from unexpected lawsuits. \ac{UL} is an independent product safety certification organization, which offers safety certifications to products \cite{UL_Web}. In order to gain the confidence of customers, the devices of the PICA system will be UL certifiable. The specific qualifications of \ac{UL} certification remain unknown to the design team, as the documents regarding the certification requirements are not publicly available. The system will also restrict \ac{EM} radiation to comply with \ac{FCC} Title 47 Part 15. It will also comply with \ac{ANSI} C12.19 and \ac{ANSI} C12.21 standards.

While these standards should ensure the general safety of the PICA devices, defects or unforeseen circumstances could imperil users or their property. The PICA system will provide a limited warranty against defects, but cannot be expected to foresee all possible circumstances. To this end, the devices will ship and work with a disclaimer regarding safe operating conditions and the hazards of tampering with the device.

In addition to ensuring the physical safety of the users, the system should also ensure the privacy and security of the users' information. While any wireless link runs the risk of packet interception and capture by a malicious observer, this will only affect the data currently being transferred, and data encryption schemes may greatly hinder these intrusions. The stored data will likely not be encrypted, but will not be actively transmitted: the only means of accessing this data will be through the software controls set in place by the base station or by physically removing the storage medium and removing the data from it. The base station software will use permissions-based file system access and will require a user to authenticate as an administrator before accessing this information. In this way, the user's data will be stored with access controls and will be kept private.

\subsubsection{Schedule Feasibility} % Fixed
As the project deadline has been fixed to be the beginning of May 2011, the timeline of the project is eight months. During these eight months, the team members must develop, prototype, test, and correct the system design. The feasibility of completing the project within this time period depends on the progress of design and the availability of parts, both of which are prone to unexpected obstacles. The design team has established a Gantt chart that establishes major project milestones and their deadlines. According to this Gantt chart, the project is currently behind schedule, as the breaker subsystem has not yet been prototyped or tested. These expectation were established as the design team will first focus on the subsystems individually and then incorporate them all into one coherent system. Although the current project status is not quite up to the expected progress, the project as a whole is still feasible within the time frame.

\subsubsection{Resource Feasibility} % Fixed
In order to meet the established goals for the project, the design team will require the major physical components of the devices, software to perform simulations, and knowledge regarding the specific devices and the design process. In general design team has found means of satisfying these requirements, whether by provision or pursuit. This does not, however, mean that no resource bottlenecks will develop.

The material components for the project will be available through an assortment of different channels. On such channel, the Calvin Engineering department, provides components that are fairly simple or common, such as solder, common circuit components, and other parts that the design team may harvest from discarded electronic devices. Calvin also provides the workspaces and equipment to perform the assembly and design. The second channel, corporate donations, provides parts to the project for little or no financial cost to the design budget through a company's academic program. Specifically, \acl{TI}donated an MSP430 development board to the project through their University Program. The final channel, the project budget, provides funds to acquire components not provided by the other channels. The Calvin Senior Design Program established a budget of \$750 for the project. Through a combination of these three means, the design team will be able to acquire the material components for the subsystem devices.

To reduce labor and resource usage, the design team should be able to test different systems without physical construction. To facilitate this, the project will require a variety of simulation software to model electronic circuits or higher-level systems. The Calvin College engineering program provides a computer and relevant simulation software to the design project, effectively meeting this need. Should the design team require any software that Calvin does not provide, the design team will attempt to work around the problem or find a cost-free program to accomplish the need. If this cannot be done, the team will investigate the availability of educational demonstration copies of the software. For the foreseeable needs of the project, the software needs remain within the available software.

Even with the necessary components and the proper tools, the design team will need the electrical and engineering knowledge required to design and implement the system. While the design team has already learned about electrical systems through the electrical engineering program at Calvin, they may consult the Calvin professors or search the Internet for information they do not have. Consumers Energy has additionally agreed to provide information and advice regarding the technical aspects of the project. In addition, the design team continues to learn how to manage themselves in an effective manner. Mark Mitchmerhuizen, a head engineer at \ac{JCI}, has volunteered to be a mentor to the design team, and can provide both technical and organizational information. Although the design team will undoubtedly encounter situations or obstacles that they cannot immediately handle, they have the ability to gain the knowledge necessary to overcome these setbacks.

Even with the knowledge required to solve problems, the design team still expects certain bottlenecks or hindrances in the design process. Time will pose the greatest obstacle to the team, as determines how much design can be accomplished before the project deadline. Additionally, ordering parts incurs some time delay between the order placement and the arrival of the part. The project budget, now \$750, exceeds some of the preliminary estimates for the project, but this increase should allow for improved flexibility in part costs. Time and money, therefore, present the two greatest bottlenecks to resource availability.

\subsection{Consumer Cost Recovery} % this is already fixed
In 2009, the residential monthly electricity bill in the United States averaged to \$104.52\cite{DOE-EIA}. If the PICA base station and solid-state breakers sell with a retail price around \$400, then the investment will amount to approximately four months of electricity bills. If, as is intended, the PICA system allows users to more wisely manage their consumption habits, then homeowners who have purchased the base station and have either the solid-state breakers or the E-meter installed at their house should experience a decrease in their monthly bill, which could recover the initial cost of the system installation. A table summarizing the different savings and payback periods appears as table \ref{tab:cost-recoup}.

\begin{table}[htbp]
 \caption{Table of Cost Recovery Rates} \label{tab:cost-recoup}
 \begin{center}
 \begin{tabular}{|c|c|c|} \hline
 \rowcolor{lightgray}
 Relative Billing Reduced & Average Monthly Savings (\$) & Months to Recover \$400 \\ \hline
 0\% & 0 & - \\ \hline
 0.5\% & 0.52 & 765 \\ \hline
 1\% & 1.05 & 382 \\ \hline
 2\% & 2.09 & 192 \\ \hline
 3\% & 3.14 & 128 \\ \hline
 5\% & 5.23 & 77 \\ \hline
 10\% & 10.45 & 38 \\ \hline
 20\% & 20.90 & 19 \\ \hline
 \end{tabular}
 \end{center}
\end{table}

 As cost reduction rates depend entirely upon the user's response to the information, the recovery period for any given customer cannot be predicted. While reduction rates of $20\%$ and higher may be realized for some individuals, a savings rate of 3\% may be somewhat more typical. Using the \ac{DOE}'s monthly average of 908 kWh consumed per residence, this represents a decrease in average consumption of about 27 kWh per month per household. This amounts to 113 watts saved for eight hours for each of thirty days per month, equivalent to slightly more than one typical incandescent lightbulb. Such a reduction yields a payback period of 128 months, or slightly less than eleven years. If, however, the system could reduce consumption by twice this amount, the payback period would halve to just over five years, or 64 months.

\subsection{Similar Products} % This has already been fixed
This section analyzes a few products with similar features to various components of the PICA system such as Kill-A-Watt, Cent-a-Meter, The Energy Dectective, and Watts Up? Smart Circuit. Finally, these products will be compared to the PICA system in a table.

\subsubsection{Kill-A-Watt} % Keep going....
Around the turn of the millennium P3 International introduced the Kill-A-Watt device, which they marketed as a "user-friendly power meter that enables people to calculate the cost to use their home appliances," \cite{About_P3}. According to Amazon.com these devices range in price from $52 to $99 Manufacturer Suggested Retail Price (MSRP) depending on features, most notably how many devices can be monitored simultaneously. P3 produces three models of the Kill-A-Watt devices:
\begin{enumerate}
\item Kill-A-Watt PS (P4320): A power strip capable of monitoring voltage, line frequency, amperage, KWH, and current leakage for up to eight devices simultaneously and includes built-in surge protection \cite{P4320_Datasheet}.
\item Kill-A-Watt (P4400): The original Kill-A-Watt device, capable of monitoring voltage, amperage, watts used, line frequency, KWH, uptime, power factor, and reactive power for 1 device \cite{P4400_Datasheet}.
\item Kill-A-Watt EZ (P4460): This device is functionally identical to the P4400 series except that it includes one extra feature, it can calculate how much a device costs the consumer, after being programmed with the \$/KWH provided by the power company \cite{P4460_Datasheet}.
\end{enumerate}
All of the Kill-A-Watt devices claim to be accurate to within 0.2\% of the actual power the monitored device uses \cite{P4320_Datasheet}\cite{P4400_Datasheet}\cite{P4460_Datasheet}. The Kill-A-Watt devices cannot replace a power meter, but simply provide a method of supplying a consumer with additional data about their power consumption.

\subsubsection{Cent-a-Meter} % Keep going....
The Australian company, Clipsal produces the Cent-a-meter also known as the Electrisave or the Owl in the UK. Clipsal only produces one version of the Cent-a-meter which displays the cost of the electricity used in the home along with the temperature and humidity \cite{Clipsal_CentAMeter}. The device can also measure kW of demand, and kg/hour of greenhouse gas emissions \cite{SmartHomeUSA}. Unlike the Kill-A-Watt, the centimeter does not accumulate any data, just displays instantaneous data on a receiver unit mounted in the home \cite{Clipsal_CentAMeter}. Clipsal does not list an MSRP for the Cent-a-meter, however SmartHome USA sells Cent-a-meter devices for \$140 \cite{SmartHomeUSA}.

\subsubsection{The Energy Detective (TED)} % Keep going....
Energy Inc., a division of 3M, recently introduced its TED (The Energy Detective) power monitor. Functionally, TED operates exactly as the meter on the exterior of a consumer's home or business but the display resides indoors in a more convenient viewing location. Energy Inc. currently produces two series of the TED device:
\begin{enumerate}
\item TED1000 series: The TED1000 devices monitor current energy consumption in killowatts, and current energy cost in \$/hour, and log this data for 13 months to predict energy use for the current billing cycle. TED1000 devices can integrate with a proprietary software package, Footprints, provided by Energy Inc. to visually display usage data \cite{TED1000}. TED1000 series devices range in price from \$119.95 to \$229.95 depending on the amp-rating of the service installation \cite{TED1000Store}.
\item TED5000 series: The TED5000 sought to improve upon the TED1000 series by extending the functionality of the TED devices. The largest selling point for the TED5000 is integration with the Google Power service to track power usage data on the web \cite{TED5000}. TED5000 series units range in price from \$239.95 to \$455.80 depending on from how many measurement units the device gathers data \cite{TED5000Store}.
\end{enumerate}

\subsubsection{Watts Up?}% Keep going....
In 1997 Electronic Educational Devices Inc. introduced the Watts Up? product line to the education market. The product immediately became a hit, and soon utility companies across the United States began to take notice \cite{WattsAbout}. EED markets the Smart Circuit devices as a replacement for traditional circuit breaker devices for 100V to 250V, 20 amp 50/60Hz circuits \cite{WattsUpDatasheet}. Each Smart Circuit contains a built in web-server that allows for aggregation of collected data at a maximum rate of once per second \cite{WattsUpDatasheet}. These Smart Circuits are typically installed into a standard panel enclosure box, similar to standard circuit breakers, mounting directly to the industry-standard DIN rail inside the enclosure\cite{WattsUpDatasheet}. Alternatively, if needed at one local outlet, the Smart Circuit can be housed in a standard double gang electrical box \cite{WattsUpDatasheet}. Each Smart Circuit device can turn itself on or off when it receives a certain remote-control signal or when it detects one of many programmable stimuli. This self-waking feature makes these devices ideal for home-automation projects \cite{WattsUpDatasheet}.

A single Smart Circuit, capable of controlling one circuit, costs \$194.95, with enclosures for one, five, or ten Smart Circuits devices going for \$325.95, \$1495.95, and \$2495.95 respectively \cite{WattsUpDatasheet}. A basic account, to view aggregated data and control the devices is free for residential use, but data rates, historical data and devices rules are limited \cite{WattsUpServices}. A top-tier account, featuring the fastest update time, 1 second, up to 25 meters, 1 year of archival data, and 25 rules costs \$50.00 a month \cite{WattsUpServices}.

\subsubsection{Smart-Watt} % Keep going....
The Smart-Watt device from Smartworks Inc. takes a similar approach to the Kill-A-Watt device in metering a single device at a time, but monitors much more information including circuit load over any period of time, and number of on/off cycles the attached device undergoes \cite{SmartWattBrochure}. The biggest advantage to the Smart-Watt devices comes from the proprietary network Smartworks has developed for their devices. Each device attaches to a local network where a central server collects and collates all the data \cite{SmartWattBrochure}. The Smart-Watt comes in two versions, one for \ac{IEC} plugs and receptacles and one for \ac{NEMA} plugs and receptacles. Both devices are similarly priced ranging from \$169 to \$249 depending on the current rating \cite{SmartWattBrochure}.

\subsubsection{Standard Power Meter} % Keep going....
Most homes or businesses attached to the electric grid are metered using a standard analogue power meter. This device provided by the power company, measures the amount of electrical energy consumed over a period of time. Typically, a power meter records in billing units, such as KWH. Each meter requires periodic readings based on the billing cycle of the power company; it is safe to assume that meters are read approximately once per month. In order to read the meter, an employee of the power company will physically go out to the meter and record usage data.

\subsubsection{Nonintrusive Appliance Load Monitor} % Keep going....
All of the products discussed here use a technique known as \ac{NILM} to monitor power consumption without affecting the load on the circuit \cite{NILM}. However, some more sophisticated products in this area use \ac{NILM} to estimate the number of individual loads on the circuit \cite{NALM}. If the research in this field proves that \ac{NILM} provides accurate and useful data, devices based on the \ac{NILM} technology would have a large advantage over other single-device power monitors, as such a device could be inserted into the feeder lines from the utility company and monitor all devices in the entire installation from a single point. Research turned up no significant products that claim to be capable of monitoring multiple loads on a circuit from a single point on the circuit. Thus this section is included to provide information, but does not represent a viable competitor in the market just yet.

\subsubsection{PICA Competitors Comparison} % Keep going....
In order to better understand the competition in the marketplace table \ref{tab:competition_sum} reproduces the information laid out above as a comparative table. The column on the far right side describes the PICA component that most directly competes with the product listed in the left column.

\begin{table}[htdp]
\caption{Comparison of PICA Competitors}
\begin{center}
\begin{tabular}{|>{\centering}b{0.75in}|>{\raggedright}b{1.25in}|>{\raggedright}b{1in}|>{\raggedright}b{0.5in}|>{\raggedright}b{0.5in}|c|}\hline
\rowcolor{lightgray}Product & Monitoring features & Control Features & Cost (Fixed) & Cost (Recurring) & PICA Competitor\\\hline
Kill-A-Watt & voltage, line frequency, amperage, KWH, and current leakage & N/A & \$52 to \$99 & N/A & Ciruit-by-Circuit Monitors\\\hline 
Cent-A- Meter & Cost of electricity, temperature, humidity, kW of demand kg/hour greenhouse gas emissions & N/A & \$140 & N/A & E-Panel Meter\\\hline 
The Energy Detective & kW load, \$/hour & N/A & \$119.95 to \$455.80 & N/A & E-Panel Meter\\\hline
Watts Up? Smart Circuit & Current, Voltage, kilowatts used & Remote On/Off & \$194.95 /circuit & Free to \$50.00 /month & Circuit-by-Circuit monitors\\\hline
Smart-Watt & voltage, line frequency, amperage, KWH, current leakage, circuit load, on/off cycles & N/A & \$169 to \$249  & N/A & Circuit-by-Circuit Monitors\\\hline
\end{tabular}
\end{center}
\label{tab:competition_sum}
\end{table}%

% Reference Nate's sections
\subsection{Production and Distribution}
See section \ref{sec:parts_and_costs}, following.

%\clearpage
%\subsection{Production} % Fixed
%As the United States contains well over 100 million homes with electrical power\cite{DOE-EIA}, even a $0.01\%$ market share represents 10,000 customers and installations. In order to produce the devices required to meet this demand, the project must recognize the fixed costs of equipment and design required to produce at all, while also approximating the cost required for each unit produced. However, these estimates only account for foreseen costs, which will almost certainly not include all the costs of designing and producing the product\cite{Nielsen_Cost_Est}.

%\subsubsection{Fixed Costs}
%The estimated fixed costs of production appear in Table \ref{tab:prod_fixed_costs}. As the operating budget for the current two-semester project amounts to \$700, the ``prototyping'' costs include both this amount and the estimated salaried value of the time the team has spent and will spend on the project. The estimate assumes that automated machinery will assemble or at least assist in assembling the devices. The certification costs represent merely an estimate, as the \ac{UL} certification costs may vary greatly depending on the length of time the process takes and the number of corrections to the product required. The facilities costs, given here as zero, indicate that the expected storage facility size will depend more on the number of units being sold, and are therefore included in the variable inventory costs. The estimated fixed costs total to \$3,240,700.

%\input{tab_prod_fixed_costs}

%\subsubsection{Variable Costs}
%The estimated variable or per-volume costs appear in \ref{tab:prod_vari_costs}. Even with the projected scale of 10,000 units, the per-unit part costs should remain fairly close to the \$50 estimate. Labor costs represent both the costs of creating and testing the devices. The shipping costs approximate the expected costs in shipping the devices for distribution, and accounts for the estimated size of the parts and the cost of shipping them on pallets. The inventory costs approximate the costs of storing the parts on pallets in a warehouse. The estimated total per-unit costs total to \$190.10.

%\input{tab_prod_vari_costs}

%\subsection{At-scale Costs}
%The average per-unit cost of production, then is a combination of the variable cost and the fixed costs distributed across every unit produced. The average cost of production for different volumes of production appears in Table \ref{tab:prod_per_unit_costs}. Assuming a baseline of 10,000 units produced, the average cost per unit should be \$517.

%\input{tab_prod_per_unit_costs}


%\subsection{Distribution} % OK...
%As the different subsystems target different markets, they also will have different means of distribution. The E-meter subsystem targets the power companies directly, so those devices will ship directly to the power companies. The other subsystems, however, cater to homeowners and others who are typically considered consumers. To distribute the appropriate subsystems to this market, the products will likely seek shelf space in home improvement and consumer electronic stores. This approach should bring the product to the attention of both types of consumers in the market: those who seek to improve their house, and those who would seek the energy-monitoring features of the PICA system.
