\documentclass[11pt,letterpaper,titlepage]{article}
% Change "article" to "report" to get rid of page number on title page
\usepackage{amsmath,amsfonts,amsthm,amssymb}
\usepackage[margin=1in]{geometry}
\usepackage{setspace}
\usepackage[T1]{fontenc}                                              %Provides T1 Font encoding for European characters
\usepackage{Tabbing}
\usepackage{array}                                                         % Allows for making new column types
\usepackage{dcolumn}					        % Provides decimal alignment column macros
\usepackage{fancyhdr}                                                  % Produce a "fancy" header using \rhead,\chead,\lhead
\usepackage{lastpage}
\usepackage{extramarks}
\usepackage{chngpage}
\usepackage{chngcntr} 						%figure and equation numbering by section
\usepackage{soul}
\usepackage{bookmark}                                                 % Prevents having to re-run due to labels
\usepackage[usenames,dvipsnames]{color}
\usepackage{graphicx,float,wrapfig}
%\usepackage{tocloft}							% Gives us the . . . in the TOC always and forever
\usepackage{ifthen}
\usepackage{listings}						% Allows for Listings
\usepackage{courier}                                                       % Provides the courier Font
\usepackage{subfig}
\usepackage{float}
%\usepackage[load-configurations = version-1]{siunitx}
\usepackage[arrowmos]{circuitikz}
%\usepackage{outlines}
\usepackage{pslatex}                                                    % This should switch the font to Times New Roman
%\usepackage[scaled=.92]{helvet}
\usepackage[printonlyused,withpage]{acronym}      % Creates acronym table of acronyms and definitions used
									        % in this document. \begin{acronym}[TDMA] ... \end{acronym}								   
\usepackage[final]{pdfpages}
%\usepackage{natbib}                                                   % Turn off for IEEE references
\usepackage{tikz-timing}
\usepackage{pdflscape}
\usepackage{multicol}
\usepackage{multirow}
\usepackage{wrapfig}
\usepackage{longtable}						% Provides the Longtable environment
\usepackage{colortbl}						% Provides \rowcolor{}
\usepackage[parfill]{parskip} 					% remove the dumb indents on all paragraphs
\usepackage[toc,page]{appendix}				% Provides more control over the Appendix
\usepackage{hyperref}						% Hyperlink output PDF Files
\usepackage[all]{hypcap}						% Fix the dumb hyperlink bug!!!

%\input{kvmacros}



%%%%%%%%%%%%%%%%%%%%%%%%%%%%%%%%%%%%%%%%%%%%%%%%
% TOC dots
%\renewcommand{\cftsecleader}{\cftdotfill{\cftdotsep}}
%%%%%%%%%%%%%%%%%%%%%%%%%%%%%%%%%%%%%%%%%%%%%%%%
% Setup the header and footer


\pagestyle{fancy}                                                       %
\fancyhead[L]{PICA LLC.}                                                 %
\fancyhead[R]{Business Plan}  %                                                   %
\fancyfoot[L]{\lastxmark}                                                      %                                                              %
\fancyfoot[C]{}
\fancyfoot[R]{Page\ \thepage\ of\ \pageref{LastPage}}                          %
\renewcommand\headrulewidth{0.4pt}                                      %
\renewcommand\footrulewidth{0.4pt}                                      %
%\renewcommand{\headheight}{14pt}
\headheight=14pt

\fancypagestyle{fancyplain}{
\fancyhead[L]{PICA LLC.}                                                 %
\fancyhead[R]{Business Plan}  %                                                 %
\fancyfoot[L]{}
\fancyfoot[C]{}                                                      %                                                              %
\fancyfoot[R]{}                          %
\renewcommand\headrulewidth{0.4pt}                                      %
\renewcommand\footrulewidth{0.4pt}                                      %
%\renewcommand{\headheight}{14pt}}
}
\definecolor{MyDarkGreen}{rgb}{0.0,0.4,0.0}
%%%%%%%%%%%%%%%%%%%%%%%%%%%%%%%%%%%%%%%%%%%%%%%%%
% Creating special column types -- mostly math types
%%%%%%%%%%%%%%%%%%%%%%%%%%%
% Start by defining some macros
\newcolumntype{d}[1]{D{.}{\cdot}{#1}}
\newcolumntype{.}{D{.}{.}{-1}}
\newcolumntype{,}{D{,}{,}{2}}
\newcolumntype{C}{>{$}c<{$}}
\newcolumntype{R}{>{$}r<{$}}
%%%%%%%%%%%%%%%%%%%%%%%%%%%%%%%%%%%%%%%%%%%%%%%%%
% For faster processing, load Matlab syntax for listings
\lstloadlanguages{MATLAB, [x86masm]Assembler, C++, VHDL}%
\lstset{frame=single,                                                                              % Single frame around code
        basicstyle=\tiny\ttfamily,             					   % Use small true type font
        keywordstyle=[1]\color{blue}\ttfamily,        				   % MATLAB functions bold and blue
        keywordstyle=[2]\color{purple},         				            % MATLAB function arguments purple
        keywordstyle=[3]\color{blue}\underbar,  			            % User functions underlined and blue
        identifierstyle=,                       						   % Nothing special about identifiers
                                                							   % Comments small dark green courier
        commentstyle=\usefont{T1}{pcr}{m}{sl}\color{MyDarkGreen}\scriptsize,
        stringstyle=\color{purple},             					   % Strings are purple
        showstringspaces=false,                 					   % Don't put marks in string spaces
        tabsize=5,                              						   % 5 spaces per tab
        %
        %%% Put standard MATLAB functions not included in the default
        %%% language here
        morekeywords={ra,stw, addi, ldw, bge, beq, br},
        %
        %%% Put MATLAB function parameters here
        morekeywords=[2]{on, off, interp},
        %
        %%% Put user defined functions here
        morekeywords=[3]{FindESS, homework_example},
        %
        morecomment=[l][\color{blue}]{...},     % Line continuation (...) like blue comment
        numbers=left,                           % Line numbers on left
        firstnumber=1,                          % Line numbers start with line 1
        numberstyle=\tiny\color{black},          % Line numbers are blue
        stepnumber=5,                            % Line numbers go in steps of 5
        breaklines=true,
        breakatwhitespace=false
        }  
 


% This is used to trace down (pin point) problems
% in latexing a document:
%\tracingall
\hfuzz4pt % I don't want to know about overfull hbox if < 2pt
%%%%%%%%%%%%%%%%%%%%%%%%%%%%%%%%%%%%%%%%%%%%%%%%%%%%%%%%%%%%%
% Some tools
\newcommand{\enterProblemHeader}[1]{\nobreak\extramarks{#1}{#1 continued on next page\ldots}\nobreak%
                                    \nobreak\extramarks{#1 (continued)}{#1 continued on next page\ldots}\nobreak}%
\newcommand{\exitProblemHeader}[1]{\nobreak\extramarks{#1 (continued)}{#1 continued on next page\ldots}\nobreak%
                                   \nobreak\extramarks{#1}{}\nobreak}%

\newlength{\labelLength}
\newcommand{\labelAnswer}[2]
  {\settowidth{\labelLength}{#1}%
   \addtolength{\labelLength}{0.25in}%
   \changetext{}{-\labelLength}{}{}{}%
   \noindent\fbox{\begin{minipage}[c]{\columnwidth}#2\end{minipage}}%
   \marginpar{\fbox{#1}}%

   % We put the blank space above in order to make sure this
   % \marginpar gets correctly placed.
   \changetext{}{+\labelLength}{}{}{}}%

\newcommand{\homeworkProblemName}{}%
\newcommand{\homeworkShortProblemName}{}
\newcounter{homeworkProblemCounter}%
\newenvironment{homeworkProblem}[2]
  {\stepcounter{homeworkProblemCounter}%
   \renewcommand{\homeworkProblemName}{#1}
   \renewcommand{\homeworkShortProblemName}{#2}
   \section*{\homeworkProblemName\ -- \homeworkShortProblemName}%
   \addcontentsline{toc}{section}{\homeworkShortProblemName}%
   \enterProblemHeader{\homeworkProblemName}}%
  {\exitProblemHeader{\homeworkProblemName}}%

\newcommand{\problemAnswer}[1]
  {\noindent\fbox{\begin{minipage}[c]{\columnwidth}#1\end{minipage}}}%

\newcommand{\problemLAnswer}[1]
  {\labelAnswer{\homeworkProblemName}{#1}}

\newcommand{\homeworkSectionName}{}%
\newlength{\homeworkSectionLabelLength}{}%
\newenvironment{homeworkSection}[1]%
  {% We put this space here to make sure we're not connected to the above.
   % Otherwise the changetext can do funny things to the other margin

   \renewcommand{\homeworkSectionName}{#1}%
   \settowidth{\homeworkSectionLabelLength}{\homeworkSectionName}%
   \addtolength{\homeworkSectionLabelLength}{0.25in}%
   \changetext{}{-\homeworkSectionLabelLength}{}{}{}%
   \subsection*{\homeworkSectionName}%
   \addcontentsline{toc}{subsection}{\homeworkSectionName}%
   \enterProblemHeader{\homeworkProblemName\ [\homeworkSectionName]}}%
  {\enterProblemHeader{\homeworkProblemName}%

   % We put the blank space above in order to make sure this margin
   % change doesn't happen too soon (otherwise \sectionAnswer's can
   % get ugly about their \marginpar placement.
   \changetext{}{+\homeworkSectionLabelLength}{}{}{}}%

\newcommand{\sectionAnswer}[1]
  {% We put this space here to make sure we're disconnected from the previous
   % passage

   \noindent\fbox{\begin{minipage}[c]{\columnwidth}#1\end{minipage}}%
   \enterProblemHeader{\homeworkProblemName}\exitProblemHeader{\homeworkProblemName}%
   \marginpar{\fbox{\homeworkSectionName}}%

   % We put the blank space above in order to make sure this
   % \marginpar gets correctly placed.
}%
   
\newcounter{subsubsubsection}[subsubsection]
\def\subsubsubsectionmark#1{}
\def\thesubsubsubsection {\thesubsubsection
     .\arabic{subsubsubsection}}
\def\subsubsubsection{\@startsection
     {subsubsubsection}{4}{\z@} {-3.25ex plus -1
     ex minus -.2ex}{1.5ex plus .2ex}{\normalsize\sf}}
% mj02r: original:
%\def\l@subsubsubsection{\@dottedtocline{4}
%     {4.8em}{4.2em}}
% mj02r: for VCE reports:
%\def\l@subsubsubsection{\@dottedtocline{4}
%     {7em}{3.8em}}
% mj02r, 29/12/2004: for thesis:
\def\l@subsubsubsection{\@dottedtocline{4}
     {11.1em}{4.6em}}

\makeatletter
\renewcommand{\paragraph}{\@startsection{paragraph}{4}{0ex}%
   {-3.25ex plus -1ex minus -0.2ex}%
   {1.5ex plus 0.2ex}%
   {\normalfont\normalsize\bfseries}}
\makeatother
   
% Includes a MATLAB script.
% The first parameter is the label, which also is the name of the script
%   without the .m.
% The second parameter is the optional caption.
\newcommand{\matlabscript}[2]
  {\begin{itemize}\item[]\lstinputlisting[language=MATLAB,caption=#2,label=#1]{#1.m}\end{itemize}}  
  
\newcommand{\ccode}[2]
  {\begin{itemize}\item[]\lstinputlisting[language={C++},caption=#2,label=#1]{#1.c}\end{itemize}}  
    
\newcommand{\assemblycode}[2]
  {\begin{itemize}\item[]\lstinputlisting[language={[x86masm]Assembler},caption=#2,label=#1]{#1.s}\end{itemize}}
  
 
\newcommand{\vhdlcode}[2]
  {\begin{itemize}\item[]\lstinputlisting[language={[AMS]VHDL},caption=#2,label=#1]{#1.vhd}\end{itemize}} 
  
\newcommand{\degree}{$^{\circ}$}
\renewcommand{\textbeta}{$\beta\ $}
\def\tm{\leavevmode\hbox{$\rm {}^{TM}$}}

%%%%%%%%%%%%%%%%%%%%%%%%%%%%%%%%%%%%%%%%%%%%%%%%%%%%%%%%%%%%%
%%%%%%%%%%%%%%%%%%%%%%%%%%%%%%%%%%%%%%%%%%%%%%%%
%This sets up the counters for the figures and equations to be numbered inside their sections
%\counterwithin{figure}{subsection}
%\counterwithin{equation}{section}
%\counterwithin{table}{subsection}
%\setcounter{secnumdepth}{5}
%\counterwithin{lst}{homeworkProblemCounter}
%%%%%%%%%%%%%%%%%%%%%%%%%%%%%%%%%%%%%%%%%%%%%%%%

%%%%%%%%%%%%%%%%%%%%%%%%%%%%%%%%%%%%%%%%%%%%%%%%%%%%%%%%%%%%%
% Make title
%\title{\vspace{2in}\textmd{\textbf{\hmwkAuthorName}}\\\normalsize\vspace{0.1in}\hmwkClass:\ \hmwkTitle \vspace{0.1in}\\Due\ on\ \hmwkDueDate\\\vspace{0.1in}\large{\textit{\hmwkClassInstructor\ \hmwkClassTime}}\vspace{3in}}
%\date{}
%\author{}
%\title{\vspace{2in}\textmd{\textbf{\hmwkClass:\ \hmwkTitle}}\\\normalsize\vspace{0.1in}\small{Due\ on\ \hmwkDueDate}\\\vspace{0.1in}\large{\textit{\hmwkClassInstructor\ \hmwkClassTime}}\vspace{3in}}
%\date{}
%\author{\textbf{\hmwkAuthorName}}

% IEEE Title
\title{Team PICA Business Plan}
\date{\today}
\author{Amy Ball, Nate Jen, Avery Sterk, Kendrick Wiersma}

%%%%%%%%%%%%%%%%%%%%%%%%%%%%%%%%%%%%%%%%%%%%%%%%%%%%%%%%%%%%%
% PDF Metadata
%\hypersetup{pdfinfo={
%                     Title={Team 01: Business Plan},
%                     Subject={Business 357},
%                     }}
%%%%%%%%%%%%%%%%%%%%%%%%%%%%%%%%%%%%%%%%%%%%%%%%%%%%%%%%%%%%%

\raggedright
\setcounter{tocdepth}{4}
\addtocontents{toc}{\protect\pagestyle{fancyplain}}
\addtocontents{lof}{\protect\pagestyle{fancyplain}}
\addtocontents{lot}{\protect\pagestyle{fancyplain}}
%\addtocontents{lol}{\protect\pagestyle{fancyplain}}

\begin{document}


%\maketitle % change this to a REAL titlepage
\begin{titlepage}
\begin{center}
\includegraphics[width=5in]{figures/TeamPicaLogo}

{\LARGE \textbf{P}ower \textbf{I}nformation \textbf{C}ollection \textbf{A}rchitecture}

\vspace{0.5in}

{\LARGE Business Plan}
\vspace{1in}

{\Large Engineering 340}

{\Large 2 March 2011}

{\Large Team 01: A. Ball, K. Wiersma, N. Jen, A. Sterk}

\vspace{1in}
\includegraphics[width=3in]{figures/Calvin_Logo}

\end{center}
\end{titlepage}
\newpage
\begin{abstract}
%The proposed business plan follows the development a power-monitoring system capable of monitoring total and circuit-by-circuit power usage for a given installation. Such a system would use non-invasive load monitoring techniques to monitor the flow of current through the feeder lines; the system can then aggregate all of this data, and perform data analysis to determine kilowatt hours used, reactive power, line frequency, and in some cases power factor. The system then can display this data to a user in real-time over a wall-mounted display unit or a web interface.

%The following report outlines the design of this system and the major decisions that the design team made. The major accomplishments thus far have been: Meeting with Consumer's Energy, Product competition research, defining a budget and per-system cost estimates, selecting features for inclusion, establishing rough requirements, investigating design possibilities, selecting a metering device for prototyping, and identifying unforeseen constraints. The remaining work for this project includes: solid-state breaker and monitor/controller design and implementation, base station design and implementation, E-meter design and implementation, component and system testing, and prototype construction.

%The final result of the project will be a working prototype which demonstrates the ability to correctly and accurately monitor power consumption. The project is scheduled to be completed by May 7, 2011.

This business plan describes the formation and operation of PICA, LLC., a company formed around the products designed by Team PICA as a senior design project at Calvin College. The company proposes to design, produce, and distribute power-monitoring equipment that may replace typical components of modern electrical building infrastructure. In particular, PICA, LLC.\ will produce a smart power meter for power companies to install on their customers' buildings and with which much more information regarding power quality can be measured, ``smart'' solid-state circuit breakers that improve user safety and deliver information about how much power each circuit in the building uses, and a base station to collect, archive, and display these measurements in real time. Armed with this much information, power consumers will be able to make wiser decisions regarding their power usage.

The market for ``smarter'' electronics and power monitoring continues to grow, and smart meters are already entering deployment on houses in select parts of the country. PICA, LLC.\ intends to produce devices that gather more information and collect and display it in a user-friendly style. The design team believes that these products are completely possible to make and distribute, and can be designed using the knowledge of electrical engineering and systems gained from Calvin's engineering program.

The team estimates first-year costs close to \$300 million, financed 60-40 from debt and equity. This means that about \$120 million will come from investors. The estimated payout time will be 5 years, by which time production and distribution means will be established and enacted.

\end{abstract}
%\setcounter{page}{0}
\tableofcontents
\thispagestyle{fancyplain}
\newpage
\thispagestyle{fancyplain}
\listoffigures
\listoftables
%\lstlistoflistings
\newpage

\protect\pagestyle{fancy}
\begin{spacing}{1.5}

%%%%%%%%%%%%%%%%%%%%%%%%%%%%%%%
% PPFS Stuff we don't use
%%%%%%%%%%%%%%%%%%%%%%%%%%%%%%%
%
%\section{Project Introduction}
\subsection{Overview of the Problem}
Standard electric meters were developed decades ago and are still used today, despite many technological advances in the last several years. Along with these technological advances, Americans have become accustomed to having access to large amounts of data, but due to the nature of the standard electric meter, data regarding the usage of power is severely limited. For the power companies, data from the meters is minimal and grid control is limited to manual operation, costing them time and money.
As the cost of electricity becomes higher and higher, electricity use in buildings is becoming a bigger concern and people have few cheap or simple ways to monitor this. Of the options available, most only address part of the whole problem, giving some information to the consumer and none to the power company or vice-versa. While there are devices such as breakers and fuses that provide electrical safety for buildings, advances in technology have made it possible to further improve safety but have not been implemented in a cost-effective way or made easily available to an average consumer, which for the purpose of this project shall be defined as a person without a mathematical or scientific education beyond high-school.

\subsection{Why the Project was Chosen}
Our team chose the project for several reasons; as future homeowners, the team has an interest in knowing more about power usage within a home. There are also many more people who would benefit from more accurate and useful data about power usage.

As good stewards of Earth we want to make sure the natural resources available are not wasted, and we believe that if there is access to more and better information, people will have a better opportunity to manage those resources more effectively. In addition, providing better information and control to the power companies can lead to less wasting of electricity on the provider's end, further contributing towards better use of Earth's resources.

Electricity-related deaths and injuries have been reduced due to devices such as fuses and breakers, but many still happen every year. As fellow human beings, we care and would like to minimize these incidents further. The technology is available and will benefit many people when implemented.

\subsection{Team Information}

Team 01, Team PICA seen in figure \ref{fig:teamphoto} consists of four engineers in Calvin College's Electrical and Computer Engineering concentration: Amy Ball, Nathan Jen, Avery Sterk, and Kendrick Wiersma.

\begin{figure}[htbp]
\begin{center}
\includegraphics[width=6in]{figures/IMG_0865}
\caption{Team PICA, left to right: Amy Ball, Kendrick Wiersma, Nate Jen, and Avery Sterk.}
\label{fig:teamphoto}
\end{center}
\end{figure}

Amy works as an intern at Johnson Controls, where she works as part of the Systems Engineering Team. She brings good communication skills, circuit-building experience, and presentation skills to the project. Her section of the project is the solid-state breakers, especially working closely with much of the analog hardware involved with the project.

Kendrick works as an intern at Raytheon Missile Systems in the Electronics Center, where he performs embedded system design and verification. Kendrick hails from Tucson, Arizona where he was born and raised. He brings real-world project experience and experience working with embedded hardware and software to the team. Kendrick leads the development of the E-meter, which measures whole-building power consumption, reporting data to the power company and the PICA base station.

Nathan has worked at Amway on the production floor and has gained involvement with club leadership at Calvin College. He brings leadership experience and a good understanding of how smaller elements of a system fit together as a whole. His section of the project is the monitoring of individual circuits and some of the control logic for the breakers.

Avery worked as an intern at the SLAC National Accelerator Laboratory doing \ac{CAD} design. He brings varied experience with software design and implementation to the project. His section of the project is the base station, especially providing the primary user interface and designing embedded software.


%\include{overview}
%\include{requirements}
%\include{designGoals}
%\include{systemDesign}
%\include{majorDesignDecisions}
%\include{eMeterDesignAlt}
%\include{breakerMonitorDesignAlt}
%\include{ssbDesignAlt}
%\include{baseStationDesignAlt}
%\include{verification}

%%%%%%%%%%%%%%%%%%%%%%%%%%%%%%%%%
% What we'll use: make sure it's all uncommented in final
%%%%%%%%%%%%%%%%%%%%%%%%%%%%%%%%%
%
\section{Mission and Vision}

\subsection{Mission}
PICA LLC seeks to help our clients better understand how their home or business consumes power through the transparent application of technology in a culturally appropriate manner. We seek to develop cost effective solutions that exhibit quality worksmanship using our expertise in Electrical and Computer Engineering.

\subsection{Vision}
PICA LLC seeks to help in building a future where power consumption is an active choice. By using modern technology to enhance power metering, we believe that a future with less dependence on fossil fuels comes one step closer to reality.

\include{industry_profile}
\include{business_strategy}
%\include{products_services} % I assume we've already provided this?
\include{marketing_strategy}
\section{Parts and Project Costs}
\label{sec:parts_and_costs}

The project costs can be broken into fixed and variable costs, where fixed costs represent the costs the team will incur during the year and production start-up costs, while variable costs represent the long-term costs associated with production.

\subsection{Fixed Costs}
\subsubsection{Prototype parts}
Through the Senior Design class, the Calvin College Engineering department provided \$750 for prototype parts. Since the project has a large scope, the team needs to minimize the cost of any single part to stay within the given budget. As such, the team chose several parts more because of their low cost rather than for functionality. As such, the team sought donations and free samples whenever possible, including the TI MSP430 development kit from Texas Instruments and the ADE7763 power monitoring chips from Analog Devices. The team also obtained a pre-purchased Xilinx Virtex-5 development board. Table \ref{tab:proto_part_cost} shows the part donations and to-date purchases. 

\begin{table}[htdp]
\caption{Prototype part costs.}
\begin{center}
\begin{tabular}{|c|c|c|c|c|c|}\hline\rowcolor{lightgray}
Date & Item & Price & Quantity & Total & Running Total\\\hline
12-Sept & ADE7763 Samples & \$0.00 & 2 & \$0.00 & \$0.00\\\hline
15-Oct & SSOP to DIP Adapter 20-Pin & \$3.95 & 2 & \$7.90 & \$7.90\\\hline
15-Oct & Break Away Headers -- Straight	& \$2.50 & 1 & \$2.50 & \$10.40\\\hline
3-Nov  & TRANSF CURRENT .50" OPENING PCB & \$14.25 & 2 & \$28.50 & \$38.90\\\hline
15-Nov & TI Donation MSP430 & \$0.00 & 1 & \$0.00 & \$38.90\\\hline
02-Feb & XO Oscillators DIP-14 3.579545M & \$1.87 \ 2 \ \$3.74 \ \$42.64 \\\hline
02-Feb & Solid State Relay 25A & \$24.07 \ 2 \ \$48.14 \ \$90.78 \\\hline
02-Feb & LCD Graphic Display Modules & \$18.55 \ 1 \ \$18.55 \ \$109.33 \\\hline
02-Feb & FFC/FPC Connectors 0.5mm & \$1.29 \ 2 \ \$2.58 \ \$111.91 \\\hline
02-Feb & Headers and Wire Housings & \$1.13 \ 2 \ \$2.26 \ \$114.17 \\\hline
02-Feb & Synchro Chron LCD and Backlight & \$22.75 \ 1 \ \$22.75 \ \$136.92 \\\hline
\end{tabular}
\end{center}
\label{tab:proto_part_cost}
\end{table}%

\subsubsection{Labor}
Throughout the semester, the team has kept a log of how many hours they worked. Determining the cost of labor for the past semester based on these records, the team can also more accurately forecast the labor cost for next semester. So far, the team has logged 348 hours, with another 40 estimated before the end of the semester. Extending this to next semester, assuming a similar workload with a little extra time for final presentations, the team expects to have 388 hours, putting the yearlong total at about 776 hours. Assuming engineers are paid \$100 an hour, the first semester labor cost is \$38,800, the second semester cost is \$38,800, making the full labor costs for the project \$77,600 as calculated in Table \ref{tab:labour_costs}. 

\begin{table}[htdp]
\caption{Team hours and projected labor costs.}
\begin{center}
\begin{tabular}{|l|c|c|}\hline\rowcolor{lightgray}
Timeframe & Hours Logged & Cost \\\hline
%start to present & 348 & 34,800\\\hline
%present to end of first semester & 40 & 4,000\\\hline
first semester & 388 & 38,800\\\hline
second semester estimate & 427 & 42,700\\\hline
TOTAL & 815 & 81,500\\\hline
\end{tabular}
\end{center}
\label{tab:labour_costs}
\end{table}%


\subsubsection{Manufacturing start-up}
In determining the cost for start-up of manufacturing, the team looked strictly at the costs of the product, ignoring many costs associated with starting a business. The team decided to contract out the work needed to build the system. Due to the high volume of systems being manufactured, the cost to manufacture will be low and is included in the cost of parts for each subsystem. The cost of constructing a storage facility is then our largest up front cost. %The team will need to determine the exact costs of this next semester, but aspects will include printing circuit boards, populating them and providing cases for each of the parts. Costs associated with buying or renting a building were broken down to a per device cost under the inventory section of the cash flow. The cost of land was determined to be irrelevant in this situation. 

%\subsubsection{Other}
%Some other costs to consider include software, discrete components, and various other development kits. In a professional engineering firm, these may represent a significant cost if they do not already own these, but as the college generously provides these, the team does not need to include them in their budget. 

\subsection{Variable Costs}
\subsubsection{Parts}
To calculate the overall cost of parts used in production, the team assumed large quantities for each of the individual components. This means that the low end of the cost range for parts is used. Based on this information and the estimated final cost of the prototype, the team calculates the cost of parts and manufacturing for the breakers, base station and e-meter to be \$35, \$100, and \$200 per subsystem, respectively.

%\subsubsection{Manufacturing}
%The manufacturing costs consist primarily of inventory and labor costs associated with production. Table \ref{tab:expected_devices} shows the calculations determining the number of devices produced in a year. Table \ref{tab:inventory_storage} shows calculations determining the amount of floor space needed to store 3 months worth of devices and the cost to add that space to a pre-existing building \cite{SteelBuilding}. 

%\begin{table}[htdp]
%\caption{Calculations showing number of devices expected.}
%\begin{center}
%\begin{tabular}{|r|p{2in}|l|}\hline\rowcolor{lightgray}
%Number    & Units                                               & Source\\\hline
%6400000  & customers                                      & Calvin Business Team\\\hline
%0.001        & percent interested in PICA         & Estimate based on \cite{Gtech_Renew}\\\hline
%64000       & customers interested in PICA    & \\\hline\hline

%25              & breakers/customer                       & Estimate\\\hline
%1600000  & breakers needed		             & \\\hline
%64000       & base stations needed		   & \\\hline\hline

%3200          & US power companies                 & \cite{EIA_Intro}\\\hline
%8300000	     & number converted homes         &\cite{Gtech_Smart_Meters}\\\hline
%24900000	     & number expected conversions & \cite{Gtech_Smart_Meters}\\\hline
%130000000  & total homes                                 & Calvin Business Team\\\hline
%40625           & average homes/company       & \\\hline\hline

%5                     & major meter providers                &\cite{Gigaom_SM_Roll}\\\hline
%3                     & minors per major                         & Estimate\\\hline
%15                   & minor meter providers                & Estimate\\\hline
%3                     & ratio homes provided to by major vs. minor & Estimate\\\hline
%30                   & effective meter providers & \\\hline\hline

%830000    & PICA homes		& \\\hline
%830000    & E-meters needed	& \\\hline	
%\end{tabular}
%\end{center}
%\label{tab:expected_devices}
%\end{table}%

%\begin{table}[htdp]
%\caption{Cost to add room to store inventory.}
%\begin{center}
%{\small
%\begin{tabular}{|>{\raggedright}p{0.5in}|C|c|c|c|c|c|c|c|}\hline\rowcolor{lightgray}
%& \mathrm{Units} & \multicolumn{2}{|>{\columncolor[gray]{0.75}}c|}{Breaker Storage} & %\multicolumn{2}{|>{\columncolor[gray]{0.75}}c|}{E-meter} & \multicolumn{2}{|>{\columncolor[gray]{0.75}}c|}{Base Station} & \\\hline

%width 	&ft		&0.0833	&given	 &0.5	                    &given	   &1	                            &given	 &\\\hline
%height	&ft		&0.25	         &given	 &0.6666	 &given	   &0.6666	&given       &\\\hline	
%length	&ft		&0.3333	&given	 &0.25	           &given       &0.3333	&given       &\\\hline	
%volume	&ft^3		&0.0069	&calculate &0.08333 &calculate &0.2222        &calculate &\\\hline\hline
%
%yr supply	&\#		&1600000	& Table \ref{tab:expected_devices}  &830000 	  & Table \ref{tab:expected_devices}  &64000   	          & Table \ref{tab:expected_devices} &\\\hline	
%qrt supply	&\#		&400000	          &calculate &207500	           &calculate  &16000  	          &calculate  &\\\hline\hline	
									
%pallet ht	&ft		&3	                   &given	   &3	                     &given	     &3	                    &given        &\\\hline	
%pallet stack&\#	&3		          &                  &3		            &                   &3                            &                 % &\\\hline		
%total ht	&ft		&9	                   &calculate  &9	                     &calculate   &9	                    &calculate  %&\\\hline\hline	
									
%floor area	&ft^2		&308.6419	&calculate   &1921.2962 &calculate  &395.0617	  &calculate  &2625\\\hline\hline
									
%floor length, if square&ft		&17.5682	& calculate	&43.8325	 &calculate	&19.8761	&calculate	&51.2347\\\hline\hline
									
%cost	&\$		          &12288	          & 20x(20+10)	&30720	&50x(50+10)	&12288	&20x(20+10)	&43000\\\hline
%\end{tabular}
%}
%\end{center}
%\label{tab:inventory_storage}
%\end{table}%



Table \ref{tab:labour_manufacture} shows the calculations used to determine the labor costs for manufacturing purposes. Estimates given during lecture \cite{Nielsen_Cost_Est} helped determine the hourly wage and additional costs of labor including insurance, vacation, holiday, sick time etc. The number of hours needed to assemble each system does not include the time needed to print and populate each circuit board as these will be completed by automatic machinery that requires minimal human interaction. 

%\begin{table}[htdp]
%\caption{Cost of labour for manufacturing}
%\begin{center}
%\begin{tabular}{|l|r|}\hline\rowcolor{lightgray}
%Line Item & Number\\\hline
%Hourly wage                                          & \$20\\\hline
%Insurance, vacation, holiday, etc.	& \$10\\\hline
%Per worker total	                            & \$30\\\hline
%Hours to assemble breakers	         & 1\\\hline
%Hours to assemble base station	& 0.5\\\hline
%Hours to assemble E-meter	         & 1.5\\\hline
%Breakers per year	                           &1,600,000\\\hline
%Base stations per year	                  & 64,000\\\hline
%E-meters per year	                           & 830,000\\\hline
%Hours for breaker	                           &1,600,000\\\hline
%Hours for base station	                  & 32,000\\\hline
%Hours for e meter	                           & 1,245,000\\\hline
%Breaker cost	                                    & \$48,000,000\\\hline
%Base station Cost	                           & \$960,000\\\hline
%E-meter cost	                                    & \$37,350,000\\\hline\hline
	 
%Total Hours	                                    & 2,877,000\\\hline
%Total Cost	                                              & \$86,310,000\\\hline
%\end{tabular}
%\end{center}
%\label{tab:labour_manufacture}
%\end{table}%

%\clearpage
%\subsubsection{Distribution}
%The team has not yet contacted a shipping company to determine exact costs, but based on experience, size and weight, the team expects a total cost of \$25,400,000. Table \ref{tab:distribution_costs} shows the cost of distribution for each~subsystem. 

%\begin{table}[htdp]
%\caption{Cost of distribution}
%\begin{center}
%\begin{tabular}{|c|r|r|r|}\hline\rowcolor{lightgray}
%	                            & Breaker	    & Base Station & E-meter\\\hline
%Number shipped	& 1,600,000 & 64,000           & 830,000\\\hline
%Cost per device         & \$5	     & \$12               & \$20          \\\hline
%Total cost	                   & \$8,000,000   & \$768,000         & \$16,600,000\\\hline
			
%Total Shipping Cost	&	\multicolumn{3}{|c|}{\$25,368,000}\\\hline
%\end{tabular}
%\end{center}
%\label{tab:distribution_costs}
%\end{table}%


\subsubsection{Marketing}
The project includes two distinct advertising methods to better accommodate the different target consumers. The e-meter aspect of the project will be sold directly to the power company, and the breakers and base station will be sold to the home and business owners. As the number of power companies is significantly fewer than the number of home and business owners, and will be purchasing in much larger quantities, the team decided it makes sense to appeal to the power companies in a much more personal manner. This includes phone calls, letters, and visits and outside of the cost of paying a few employees will be negligible. 

Most of the advertising will aim at the home and business owners, and the team decided that magazines and websites such as Popular Science and Green magazine are the best method of reaching out to potential buyers.  Green magazine features news and products related to sustainable energy, reaching thousands of people every year. Approximately 36\% of those people are in the building and contracting industry and would be beneficial in spreading news about the team's product \cite{GreenMediaKit}. The cost to put a medium size ad on their website is \$150 dollars per month \cite{GreenMediaKit}, so for a year would be \$1800. Popular science reaches over 7 million people using printed material. For a 1/3 page ad in four color for 12 months, the cost is \$59,900 \cite{PopSci}. The team would like to target 3 to 4 magazines and using Popular Science and Green magazine as boundary cases, estimates a total cost of \$120,000 for marketing and advertising. 

For the home and business owners' side of the project, the team also would like to work with distributers like Lowe's and Home Depot. The team would like to use a method of advertising similar to the one used for power companies, so the cost will not noticeably increase. The distribution companies may do additional advertising, but any costs associated with that will be their responsibility, so again the cost the design team expects will stay the same.

\subsubsection{Legal, warranty and support}
The team  expects about 10 hours of work for basic legal documentation. Because of the potential for lawsuits, the team built in money to cover the costs of 200 hours of work, assuming \$80 an hour, giving a total of \$16,800. The team does not intend to pursue any patents, but recognizes there may be infringement lawsuits, which were built into the above 200 hours.

The team expects 5\% of all PICA systems that include all three subsystems to fail and need replacement. At a system cost of \$260 per system with shipping of \$30, the amount needed to cover warranties is \$2,432,000.
%\subsection{Total Costs}
%Table \ref{ProjectCosts.tex} is a summary of the project costs, including both fixed and variable costs.

%\input{ProjectCosts}

%\section{Location and Layout}
\subsection{Initial Location and Layout}
To start, the company will have one location -- a 6,000 square foot store located conveniently close to the downtown area of Grand Rapids, Michigan. It includes a development area, service department, offices, and a showroom area.  

\subsection{Geographic Expansion}
Today, many products that are manufactured in one state could be used in other states as well; however, it is always helpful to have a closer location to the customers to offer better support, so successfully choosing the location is very important. Once the Grand Rapids location grows, the company will expand to other parts of Michigan along with a few major cities all around the United States. The first city the company looks to expand to is Chicago, given its close proximity to the headquarters in Michigan. After proven growth in Chicago, the company would expand to other big cities in the west and south, including Los Angeles, CA or Dallas, TX. 

              
\include{competitor_analysis} % provides competitor analysis
%\section{Project Management}
\subsection{Team Organization}
See figure \ref{fig:orgchart} for the team organizational chart.

\begin{figure}[htbp]
\begin{center}
\includegraphics[width=6.5in]{figures/TeamOrgChart}
\caption{Team PICA Organizational chart.}
\label{fig:orgchart}
\end{center}
\end{figure}

\subsection{Team Responsibilities}
\subsubsection{Amy}
Amy, along with Nathan, makes up the hardware part of the team. They are in charge making decisions regarding hardware-specific sections of the project. Amy is also in charge of the Breaker sub-system of the project. She has the most complete understanding of their functions and specific requirements, she may delegate sub-sections to other team members, but she will have a good grasp of how they fit into the larger system. A sub-system of the Breakers is the monitoring section, which the team delegates to Nathan. Nathan and Amy will be working together closely with the full sub-system of the solid-state breakers.

\subsubsection{Nathan}
Nathan is co-leader with Kendrick. They are in charge of scheduling and assigning tasks, scheduling meetings, keeping team on time, on task, and will answer project questions. Nathan, along with Amy, makes up the hardware part of the team. They are in charge making decisions regarding hardware-specific sections of the project. Nathan is also in charge of helping integrate the sub-sections into the whole system, which includes having a detailed, basic understanding of each sub-system, and knowing how each relates to the overall requirements and goals. Nathan is also in charge of the monitoring section of the solid-state breakers.

\subsubsection{Avery}
Avery is in charge of the Base Station section of the project. He has the most complete understanding of it's function and specific requirements. He may delegate sub-sections to other team members, but he will have a good grasp of how they fit into the larger system. He, along with Kendrick, makes up the software part of the team; they make decisions regarding software-heavy sections of the project.

\subsubsection{Kendrick}
Kendrick is in charge of the E-Panel section of the project. He has the most complete understanding of it's function and specific requirements. He may delegate sub-sections to other team members, but he will have a good grasp of how they fit into the larger system. He is also co-leader with Nathan and they will share duties as necessary. He, along with Avery, makes up the software part of the team; they make decisions regarding software-heavy sections of the project.

\subsection{Schedule}
In order to complete the project within the established deadlines, the design team established a schedule and task-oriented deadlines to supplement the larger senior design program-established deadlines. In doing so, the design team selected to address the subsystems individually. The design team will first focus on the solid-state circuit breakers and circuit-monitoring devices, which they hope to complete by the end of the fall semester. The other two subsystems, the base station and the main smart meter, will gain focus in the second semester. Table \ref{Task_list.tex} shows a list of major milestones for the project with the estimated date of completion for each. The chart also shows the number of hours estimated needed for each task and any dependencies. Items in bold indicate tasks scheduled for completion by the current date and labor totals are shown in the bottom right. 

{
\small
\begin{longtable}[c]{|>{\raggedright}b{2in}|>{\raggedright}b{1in}|>{\raggedright}b{1in}|b{0.75in}|b{1in}|}
\caption{Task lists\label{Task_list.tex}}\\
\hline
\rowcolor{lightgray}
Milestone & Estimated Completion & Dependence & Estimated Labor &  \\
\hline
\endfirsthead
\caption[]{Continued from previous page}\\

\hline
\rowcolor{lightgray}
Milestone & Estimated Completion & Dependence & Estimated Labor &  \\
\hline
\endhead
\multicolumn{5}{r}{{Continued on next page}} \\
\endfoot

\endlastfoot
Project scope and functionality determined & Oct 25, 2010                                & none                      & 40  &              \\\hline
Project goals and requirements set         & Nov 16, 2010                                & Scope and functionality   & 60  &              \\\hline
Design criteria determined                 & Nov 19, 2010                                & Goals and requirements    & 15  &              \\\hline
Preliminary breaker design set             & Nov 23, 2010                                & Criteria                  & 50  &              \\\hline
Preliminary e-meter design set             & Feb 2, 2010                                 & Criteria                  & 50  &              \\\hline
Preliminary base station design set        & Mar 6, 2010                                 & Criteria                  & 50  &              \\\hline
Breaker prototyped and tested              & Dec 1, 2010                                 & Design set                & 30  &              \\\hline
E-meter prototyped and tested              & Feb 13, 2010                                & Design set                & 30  &              \\\hline
Base station prototyped and tested         & Mar 19, 2010                                & Design set                & 30  &              \\\hline
Full system integration                    & Apr 17, 2010                                & All subsystems prototyped & 40  &              \\\hline
PPFS turned in                             & Dec 7, 2010                                 & Criteria                  & 130 &              \\\hline
Project presented                          & May 16, 2010                                & Prototyped                & 40  &              \\\hline
                                           &                                             &                           & 325 & 1st Semester \\\hline
                                           &                                             &                           & 240 & 2nd Semester \\\hline
                                           &                                             &                           & 565 & Total        \\\hline
\end{longtable}
}



While subsystems should be able to be completed independent of each other, the design team decided to address the circuit-by-circuit monitoring component of the project first because of its application to control systems. As members of the design team are taking a class in control systems during the fall semester, working on the breakers while also learning control theory seemed to create an advantageous symbiosis between working and learning. The control systems course also seeks to use aspects of the senior design project as learning experiences, so the motivation to combine the controls assignment and the breaker modules is twofold.

The design team has assembled its internal schedule into Gantt chart to show the deadlines of tasks and the linkages between different tasks. Despite having this planning tool and list of tasks, the actual flow and completion of work is frequently different from originally expected. This is partially due to unexpected emergence of assignments and deadlines for the design project itself, but other classes also contribute unforeseen and time-consuming work that impedes progress in the project. To deal with outside deadlines and work, the team tried to think of deadlines focus on work and deadlines to two or three weeks out. Team members were encouraged to think in general terms only about deadlines more than a couple weeks out to help keep focus on current progress. As the semester progresses, these emergent due-dates should decrease in number and severity, allowing the design team to devote more time to the project.

In order to complete the project, the team needs to accomplish a variety of tasks shown in the work breakdown structure below. Much of the early work includes a lot of paperwork outside of the actual design of the system. This includes determining exactly what problem the team is addressing and the requirements needed to solve it. Project goals and design criteria also make up some of the pre-design work. In industry, the customer would already have set many of these requirements, goals and criteria, so to make the project as `real-world-like' as possible, the team put these before the design stage of the project. To make the requirements and goals realistic, the team decided to meet with a variety of professionals in fields including marketing, business and engineering in addition to potential customers. 

Once the project goals, requirements and criteria are set, the team can begin some preliminary design. As another way to limit the project and keep it from getting out of control, the project scope needs to be determined early in the project. A big factor in setting the scope is making the system unique from products that solve similar problems. The first step the team would like to take is to determine the basic functionality of both systems and subsystems and set up functional block diagrams to show this. After the basic functions of each of the subsystems are determined, the team will explore various solutions through general research and trade studies. When the team has compiled a number of different solutions, they can start to eliminate possibilities based on criteria established earlier. To assist with this, the team will use design matrices and comparison tables as visual aids.

After narrowing down the possible solutions, the team will begin the design and implementation stage. Further criteria may include power efficiency, cost of building the system, and availability of components. As  design aspects apply to all subsystems, and since development of the subsystems will occur at different rates as mentioned earlier, the finalized designs will not finish simultaneously. Testing will occur after designing and building the various parts of the subsystems. The team decided that the breakers, breaker monitors, main system monitor and wireless communication should be the first parts of the system completed. Other parts that will require finalization and building include the upgrade mechanism, base station, display module and firmware. Each of the parts will require modifications based on the testing results. Modifications to each of the parts may be necessary based on the testing results, so this section may take multiple iterations before it is complete.

Assembly of the full system will follow the individual parts' completion, along with testing for a variety of things. Some specific things the team hopes to include in the testing are tests based on outside criteria, tests based on requirements set by the team and testing common possible failure points. As the team is working with a limited amount of time that may not allow for full testing of the system, some of the long term testing procedures including accelerated lifetime testing will be written up for later use outside of the class.

Communication will be critical to the success of the project, both during and after so the team aims to provide documentation of all design decisions and testing done throughout the year. As the customer will need to interact with the system effectively, the team hopes to provide a user manual as well. For the class, the team needs to present the project on a number of occasions and hopes to demonstrate their work to all groups associated with the project, including business groups, customers and technical professionals.


  %includes scheduling crap that Nate didn't want to deal with
\documentclass[11pt,letterpaper]{article}

\usepackage[utf8]{inputenc}
%%% PACKAGES
\usepackage{array}
\usepackage{verbatim}
\usepackage{subfig}
\usepackage[margin=1in]{geometry}
\usepackage{graphicx}

\usepackage{hyperref}
\usepackage{hypcap}

\title{Business Organization}
\author{PICA, LLC.}
%\date{} % Activate to display a given date or no date (if empty),
         % otherwise the current date is printed 

\begin{document}
\maketitle

\section{Plan of Operation}


\subsection{Legal form of ownership}
The business shall be formed as a limited liability company under the name "PICA, LLC."

\subsection{Company structure}

\begin{figure}[htbp]
 \begin{center}
  \includegraphics[trim=1.5in 1.75in 3in 0.75in,clip,width=0.6\textwidth]{includes/PICA_LLC_Org_Chart}
 \end{center}
 \caption{Proposed Organization Chart for PICA, LLC.}
 \label{fig:org-chart}
\end{figure}

\subsubsection{Select explanations of Figure \ref{fig:org-chart} }
The Internal Affairs directorate shall provide certain services to the other directorates and departmetns. For example, Internal Affairs includes the Human Resources division, but the other directorates will of course require the services of the HR division in order to hire new employees. Similarly, the finances division will provide budgeting information to the other divisions and directorates.

The other two directorates do not provide services to the other directorates. The one exception to this is the Observer function of the External Affairs directorate: its purpose is to monitor the business and legal climate in terms to demands and regulations, then inform the appropriate divisions. This could be spun off into its own directorate, but will likely not employ enough people to grant it a director and a direct voice in the Circle of Directors.

\subsection{Decision-making authority chain}
The final authority on a decision shall be vested in the Circle of Directors, but their authority shall not be required in every decision. The Circle of Directors shall have the exclusive power to make decisions regarding the direction of the business and the relationships between the directorates. Other issues may rise to the Circle if they cannot be resolved at a lower level or if the scope of the decision cannot be contained to one directorate. Otherwise, decisions that are limited in scope to any particular group or body shall be resolved within that body with the advice of the applicable Internal Affairs departments.



\subsection{Compensation/benefits package}
All employees shall be compensated fairly and in proportion to the scope of their decisions and actions. The company shall provide benefits packages including medical- prescription, vision, and dental plans. In addition, the company shall also provide a cafeteria and snacks for its employees to enjoy within moderation. The lowest Engineering position shall pay an annual salary of \$55,000, with up to 50\% increases in pay for each level of management ascended. Employees in other directorates will earn an annual salary of \$35,000, with a similar reward for ascending management. Employees shall receive raises in pay for distinguishing their work from that of their peers.


\end{document}

%\section{Financial Forecasts}
\subsection{E-Meter}
\input{tables/EMeter/IncomeStatement}
\input{tables/EMeter/CashFlow}
\input{tables/EMeter/BreakEvenAnalysis}

\subsection{Base Station}

\input{tables/BaseStation/IncomeStatement}
\input{tables/BaseStation/CashFlow}
\input{tables/BaseStation/BreakEvenAnalysis}

\subsection{Solid State Breaker}

\input{tables/Breaker/IncomeStatement}
\input{tables/Breaker/CashFlow}
\input{tables/Breaker/BreakEvenAnalysis}
\newpage  % old section
\include{ten_year_forecast} % new section


\section{Loan or Investment Proposal}

\subsection{Amount Requested -- Equity and/or Debt}
%3.5M\$
In order to successfully enter the smart metering market, PICA LLC feels that they will require \$290 million in assets. In order to have all these assets, we plan on using a 60-40 split for debt and equity. Thus, our debt-equity structure for startup becomes \$174 million in debt and \$116 million in equity. We hope to raise the \$116 million through personal donations from venture capitalists while the remainder of the cash supply comes from a business loan.

\subsection{Purpose and Uses of Funds}
We plan to initially contract for the fabrication of the circuit boards until we gain enough operating capital to move production in-house. Our startup costs represent initial cost of parts to begin production and salary for our employees during the first year. Until our business proves to be successful we plan on renting office and warehouse space for the first year. After the first year, assuming financial success, we hope to build our own warehouse to store completed systems before shipment to retailers.

\subsection{Repayment and Cash Out Schedule}
By year 5 of business, PICA LLC forecasts enough financial success to being repaying our venture capitalists. The return on investment for these venture capitalists, assuming economic success will be 8-10\% of the initial investment.

\subsection{Timetable for Implementing and Launching a Business}
PICA LLC projects a 6-month timeframe for starting the business. During these 6-months, PICA LLC intends to finish and prove the initial prototype and get all necessary contracts in place for office space, warehouse storage, shipping, and production facilities. After these are in place, PICA LLC projects that by the end of year 1, they will be shipping completed products to retail locations.


%\include{acknowledgements}
%\include{conclusions}
\include{bibliography}

\begin{appendix}
%\begin{table}[htdp]
%\caption{default}
\section{Acronyms}
\begin{acronym}[TDMA]
\acro{AC}{Alternating Current}
\acro{AD}{Analog Devices}
\acro{ADC}{Analog-to-Digital Converter}
\acro{AES}{Advanced Encryption Standard}
\acro{AMR}{Automated Meter Reading}
\acro{ANSI}{American National Standards Institute}
\acro{CAD}{Computer Aided Design}
\acro{CDMA}{Code Devision Multiple Access}
\acro{CF}{Compact Flash}
\acro{DC}{Direct Current}
\acro{DHCP}{Dynamic Host Configuration Protocol}
\acro{DOE}{United States Department of Energy}
\acro{EEPROM}{Electronically Erasable Programable Read Only Memory}
\acro{ESP}{Electronic Signal Processing}
\acro{EM}{electromagnetic}
\acro{FCC}{Federal Communications Commission}
\acro{FET}{Field-Effect Transistors}
\acro{FPGA}{Field Programable Gate Array}
\acro{GPL}{GNU General Public License}
\acro{HTTP}{Hypertext Transfer Protocol}
\acro{IC}{Integrated Circuit}
\acro{IEC}{International Electrotechnical Commision}
\acro{IEEE}{International Electrical and Electronics Engineers}
\acro{JCI}{Johnson Controls, Inc.}
\acro{LAN}{Local Area Network}
\acro{LCD}{Liquid Crystal Display}
\acro{LED}{Light Emitting Diode}
\acro{MCU}{Master Control Unit}
\acro{MSRP}{Manufacturer Suggested Retail Price}
\acro{NEMA}{National Electrical Manufacturers Association}
\acro{NIC}{Network Interface Card}
\acro{NILM}{Non-intrusive Load Monitoring}
\acro{NPC}{National Power Corporation}
\acro{NTP}{Network Time Protocol}
\acro{OS}{Operating System}
\acro{PC}{Personal Computer}
\acro{RAM}{Random-Access Memory}
\acro{RFID}{Radio Frequency Identification Device}
\acro{RMS}{Root Mean Square}
\acro{SD}{Secure Digital}
\acro{SoC}{System on a Chip}
\acro{SPI}{Serial Peripheral Bus}
\acro{TI}{Texas Instruments}
\acro{UL}{Underwriters Laboratories}
\end{acronym}
%\label{default}
%\end{table}%

%\section{Team Information}

Team 01, Team PICA seen in figure \ref{fig:teamphoto} consists of four engineers in Calvin College's Electrical and Computer Engineering concentration: Amy Ball, Nathan Jen, Avery Sterk, and Kendrick Wiersma.

\begin{figure}[htbp]
\begin{center}
\includegraphics[width=6in]{figures/IMG_0865}
\caption{Team PICA, left to right: Amy Ball, Kendrick Wiersma, Nate Jen, and Avery Sterk.}
\label{fig:teamphoto}
\end{center}
\end{figure}

Amy works as an intern at Johnson Controls, where she works as part of the Systems Engineering Team. She brings good communication skills, circuit-building experience, and presentation skills to the project. Her section of the project is the solid-state breakers, especially working closely with much of the analog hardware involved with the project.

Kendrick works as an intern at Raytheon Missile Systems in the Electronics Center, where he performs embedded system design and verification. Kendrick hails from Tucson, Arizona where he was born and raised. He brings real-world project experience and experience working with embedded hardware and software to the team. Kendrick leads the development of the E-meter, which measures whole-building power consumption, reporting data to the power company and the PICA base station.

Nathan has worked at Amway on the production floor and has gained involvement with club leadership at Calvin College. He brings leadership experience and a good understanding of how smaller elements of a system fit together as a whole. His section of the project is the monitoring of individual circuits and some of the control logic for the breakers.

Avery worked as an intern at the SLAC National Accelerator Laboratory doing \ac{CAD} design. He brings varied experience with software design and implementation to the project. His section of the project is the base station, especially providing the primary user interface and designing embedded software.



%\section{Team Resumes}
%The following pages contain each team-member's resume.

%\includepdf[pages={1},pagecommand={},scale=.9]{resume/Ball_Amy}
%\includepdf[pages={1},pagecommand={},scale=.9]{resume/Wiersma_Kendrick}
%\includepdf[pages={1},pagecommand={},scale=.9]{resume/Jen_Nathan}
%\includepdf[pages={1},pagecommand={},scale=.9]{resume/Sterk_Avery} 
\end{appendix}

\end{spacing}
\end{document}

%%%%%%%%%%%%%%%%%%%%%%%%%%%%%%%%%%%%%%%%%%%%%%%%%%%%%%%%%%%%%
