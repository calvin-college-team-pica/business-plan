\section{Loan or Investment Proposal}

\subsection{Amount Requested -- Equity and/or Debt}
%3.5M\$
In order to successfully enter the smart metering market, PICA LLC feels that they will require \$290 million in assets. In order to have all these assets, we plan on using a 60-40 split for debt and equity. Thus, our debt-equity structure for startup becomes \$174 million in debt and \$116 million in equity. We hope to raise the \$116 million through personal donations from venture capitalists while the remainder of the cash supply comes from a business loan.

\subsection{Purpose and Uses of Funds}
We plan to initially contract for the fabrication of the circuit boards until we gain enough operating capital to move production in-house. Our startup costs represent initial cost of parts to begin production and salary for our employees during the first year. Until our business proves to be successful we plan on renting office and warehouse space for the first year. After the first year, assuming financial success, we hope to build our own warehouse to store completed systems before shipment to retailers.

\subsection{Repayment and Cash Out Schedule}
By year 5 of business, PICA LLC forecasts enough financial success to being repaying our venture capitalists. The return on investment for these venture capitalists, assuming economic success will be 8-10\% of the initial investment.

\subsection{Timetable for Implementing and Launching a Business}
PICA LLC projects a 6-month timeframe for starting the business. During these 6-months, PICA LLC intends to finish and prove the initial prototype and get all necessary contracts in place for office space, warehouse storage, shipping, and production facilities. After these are in place, PICA LLC projects that by the end of year 1, they will be shipping completed products to retail locations.
